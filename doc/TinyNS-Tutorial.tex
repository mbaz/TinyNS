\documentclass[11pt]{report}

% source code highligthing
\usepackage[chapter]{minted}
\newminted{python}{autogobble,linenos,fontsize=\footnotesize}

% configure page
\usepackage{geometry}
\geometry{paper=letterpaper,tmargin=1in,lmargin=1.3in,rmargin=1.3in}

\hyphenpenalty 100

% use my math macros
\usepackage{mbmath}
% use libertine text with newtxmath
\usepackage{mbfontslib}
% use my "units" package
\usepackage{mbunits}

\usepackage{url}
\usepackage{booktabs}
\usepackage{polyglossia}

% configure section titles
\usepackage{sectsty}
\allsectionsfont{\sffamily\large}

% configure figures
\usepackage{tikz}
\usetikzlibrary{positioning}
\usepackage{pgfplots}
\pgfplotsset{compat=newest}
\pgfplotsset{plot coordinates/math parser=false}

\begin{document}
\title{TinyNS Tutorial}
\author{Miguel Bazdresch}
\date{\today}

\maketitle
\thispagestyle{empty}

\section*{Introduction}

TinyTS is a very small and simple network simulator. The main purpose of TinyTS
is to enable you to implement, test and evaluate your own routing
algorithms in a variety of network topologies. Although this can be done in a
larger, professional simulator such as NS3, TinyNS has a number of features
that allow you to focus on the fundamentals, instead of spending your
time learning to use the tool:

\begin{enumerate}
	\item In its tiniest form, it is only 149 lines long. This means that you
		can understand the simulator inside and out, which makes it easier to
		modify it.
	\item It is written in Python, which is a very accessible language.
	\item It does not use discrete events. While this limits its
		power and flexibility, it results in a design that is very simple and
		easy to understand.
\end{enumerate}

To be sure, I believe you should eventually move on to more powerful
tools such as NS3, OpNet, etcetera. However, a first network simulation
experience with TinyNS can be a solid foundation on which to build on.

\section*{The simulator}

There are a few core ideas behind the simulator:

\begin{enumerate}
	\item There are two kinds of nodes: edge nodes and switches. An edge node
		generates and consumes packets; a switch forwards packets.
	\item Edge nodes have only one port; switches can have any number of ports.
	\item Nodes are connected by connecting their ports. Ports are
		bidirectional.
	\item The simulation consists of a repetition of these three steps:
		\begin{enumerate}
			\item Edge nodes are allowed to receive one packet.
			\item Switches forward one packet stored in their memory.
			\item Edge nodes are allowed to transmit a packet.
		\end{enumerate}
\end{enumerate}

Let us look at the code that implements these ideas.

\subsection*{Packets}

A packet is implemented as follows:

\begin{minted}{python}
class Packet:
    def __init__(self, s_addr, d_addr, payload):
        self.s_addr = s_addr
        self.d_addr = d_addr
        self.payload = payload
\end{minted}

The meaning of the variables is as follows:
\begin{itemize}
	\item \mintinline{python}{s_addr} is the address of the source, the packet originator.
	\item \mintinline{python}{d_addr} is the address of the destination.
	\item \mintinline{python}{payload} is the actual data carried by the
		packet.
\end{itemize}

\subsection*{Edge nodes}

An edge node is defined as follows:
\begin{minted}{python}
class EdgeNode:
    def __init__(self, address, d_addr_set, tx_prob):
        self.address = address
        self.d_addr_set = d_addr_set
        self.tx_prob = tx_prob
        self.memory = []
\end{minted}

The fields have the following meaning:
\begin{itemize}
	\item \mintinline{python}{address} is the node's address. Packets meant for
		this node need to put this address as destination.
	\item \mintinline{python}{d_addr_set} is a list of the possible
		destinations this node may send packets to. The node will choose among
		this list at random.
	\item \mintinline{python}{tx_prob} is the probability that a node actually
		generates a packet when given the opportunity.
	\item \mintinline{python}{memory} The node's buffer (actually a FIFO).
		Arriving packets will be placed at the end of the buffer.
\end{itemize}

A node has two functions to help the simulator tell it apart from a switch:

\begin{minted}{python}
def isedge(self):
    return True

def isswitch(self):
    return False
\end{minted}

The edge node class has a function to transmit a packet. The function is called
with the current simulation time for logging purposes. A uniform-distributed
random number is generated, and if its value is less than
\mintinline{python}{tx_prob}, then a packet is generated and returned to the
simulator. If no packet is transmitted, then the node returns
\mintinline{python}{None}.

\begin{minted}{python}
def transmit(self, time):
    # Determine if a packet will be generated
    if random() < self.tx_prob:
        # Determine destination address
        d_addr = choice(self.d_addr_set)
        # instantiate packet
        pkt = Packet(self.address, d_addr, 'dummy payload')
        self.txaction(pkt, time)
        return pkt
    else:
        return None
\end{minted}

The edge nodes also have a function to receive a packet.

\begin{minted}{python}
def receive(self, time):
    if len(self.memory) > 0:
        pkt = self.memory.pop()
        self.rxaction(pkt, time)
\end{minted}

Note that these last two functions call two other auxiliary functions.
\mintinline{python}{transmit} calls \mintinline{python}{txaction}, and
\mintinline{python}{receive} calls \mintinline{python}{rxaction}. This allows
you to specify some additional actions that the edge node will take when
transmitting and receiving packets. One example is logging (either to the
screen or to a file). More advanced networks may require additional actions.
Note that these functions do nothing in the minimal TinyNS implementation.

\subsection*{Switch nodes}

A switch node never creates its own packets. All it does is forward packets
received through one port through a different port. The output port will
hopefully be chosen so that the packet gets closer to its destination.

A switch can have any number of ports. However, the switch can only forward a
single packet at a time.

The switch uses a routing table to decide how to forward the packets. An
initial routing table needs to be provided when instantiating a switch;
however, this table need to be optimal or even correct.

A switch node is defined as follows:

\begin{minted}{python}
class SwitchNode:
    # Class initializer
    def __init__(self, ID, ports, rt):
        self.ID = ID
        self.ports = ports
        self.rt = rt
        self.memory = []
\end{minted}

The fields have the following meaning:

\begin{itemize}
	\item \mintinline{python}{ID} is the switch's unique ID.
	\item \mintinline{python}{ports} is a list of the switch's ports.
	\item \mintinline{python}{rt} is the switch's routing table, implemented as
		a dictionary.
	\item \mintinline{python}{memory} is the node's buffer memory.
\end{itemize}

Two functions let the simulator tell an edge node apart from a switch:

\begin{minted}{python}
def isedge(self):
    return False

def isswitch(self):
    return True
\end{minted}

Packets are forwarded with the following function. If there is a packet in the
buffer, then the packet is removed from the buffer and its destination address
is obtained. The routing table is consulted to find the port the packet should
be forwarded to. If the buffer is empty, then a tuple
\mintinline{python}{(None, None)} is returned.

\begin{minted}{python}
def forward(self, time):
    # if there is something in memory, process it
    if len(self.memory) > 0:
        pkt = self.memory.pop()
        # determine output port
        outport = None
        for i in self.rt:
            if i == pkt.d_addr:
                outport = self.rt[i]
        # execute an action
        self.action(pkt, outport)
        # return packet and output port
        return pkt, outport
    else:
        return None, None
\end{minted}

Note that, just like with the edge nodes, an action function is invoked when a
packet is forwarded. In this minimal implementation, that function does
nothing.

\section*{The simulator control}

The simulation executes what is called a ``round robin scheduler'', which calls
each node in order and asks it to perform a function. The key function is
called \mintinline{python}{run}. It first lets edge nodes receive; then, it
calls the switches to forward one packet; finally, it lets the edge nodes
transmit. This is executed for the desired number of iterations. This function
uses the node list and the connection list to know which nodes to call, and in
which node's memory to write the packets as they flow through the network.

\begin{minted}{python}
def run(nodelist, connlist, iterations):
    for i in range(iterations):
        # Edge nodes receive
        for n in nodelist.nodelist:
            if n.isedge():
                n.receive(i)
        # Switch nodes forward
        for n in nodelist.nodelist:
            if n.isswitch():
                pkt, port = n.forward(i)
                if pkt != None:
                    nextnode = findnextnode(connlist, n, port)
        # Edge nodes transmit
        for n in nodelist.nodelist:
            if n.isedge():
                pkt = n.transmit(i)
                if pkt != None:
                    nextnode = findnextnode(connlist, n, '1')
                    nextnode.memory.insert(0,pkt)
                    nextnode.memory.insert(0,pkt)
    print("Done!")
\end{minted}

\subsection*{Creating the network}

To create a network, you need to instantiate the nodes and connect them. To do
that, you'll need a node list and a connection list:

\begin{minted}{python}
# List of connections
class ConnectionList:
    def __init__(self):
        self.connlist = []

    def connect(self, node1, port1, node2, port2):
        self.connlist.append((node1, port1, node2, port2))

# List of nodes
class NodeList:
    def __init__(self):
        self.nodelist = []

    def append_node(self, node):
        self.nodelist.append(node)
\end{minted}

\section*{Putting it all together: an example}

Let's say that we want to simulate the following network:

\begin{tikzpicture}
\node (s) [draw,circle] {S};
\node (e1) [draw,circle,below=of s] {E1};
\node (e2) [draw,circle,above=of s] {E2};
\node (e3) [draw,circle,left=of s] {E3};
\draw[<->,shorten <=1pt, shorten >=1pt] (s) -- node[left,pos=0.2] {1} (e1);
\draw[<->,shorten <=1pt, shorten >=1pt] (s) -- node[right,pos=0.2] {2} (e2);
\draw[<->,shorten <=1pt, shorten >=1pt] (s) -- node[above,pos=0.2] {3} (e3);
\end{tikzpicture}

Furthermore, we want each edge node to transmit a different message. We need to
follow these steps:

\begin{enumerate}
	\item Create the node list and connection list.
\begin{minted}{python}
nl = NodeList()
cl = ConnectionList()
\end{minted}
	\item Create the edge nodes and append them to the node list.
\begin{minted}{python}
e1 = EdgeNode('E1',['E2'],1)
e2 = EdgeNode('E2',['E1','E3'],1)
e3 = EdgeNode('E3',['E1','E2'],1)
nl.append_node(e1)
nl.append_node(e2)
nl.append_node(e3)
\end{minted}
	\item Create the switch and append it to the node list.
\begin{minted}{python}
s = SwitchNode('S',['1','2','3'],dict([('E1','1'),('E2','2'),('E3','3')]))
nl.append_node(s)
\end{minted}
	\item Connect the nodes:
\begin{minted}{python}
cl.connect(e1,'1',s,'1')
cl.connect(e2,'1',s,'2')
cl.connect(e3,'1',s,'3')
\end{minted}
	\item Run the simulation for the desired number of steps:
\begin{minted}{python}
run(nl,cl,10)
\end{minted}
\end{enumerate}

The code to run this simple example can be found in
\mintinline{python}{test/simple.py}.

\end{document}
